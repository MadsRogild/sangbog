\documentclass[a4paper,12pt,final,twoside,openany,article]{memoir}%,oneside

\settypeblocksize{0.85\paperheight}{0.9\paperwidth}{*}
%One-sided
%\setlrmargins{*}{*}{1}
%Two-sided
\setlrmargins{*}{*}{.66}

\setulmargins{*}{*}{*}
%\setheadfoot{\onelineskip}{0pt}
\setheadfoot{0pt}{\onelineskip}
\checkandfixthelayout

%\OnehalfSpacing

%\setsecnumdepth{subsection}
%\chapterstyle{tandh}
%\chapterstyle{section}
%\renewcommand{\chaptitlefont}{\bfseries\Large}
%\renewcommand{\chapnumfont}{\bfseries\Large}
\chapterstyle{article}


\usepackage[utf8x]{inputenc}
\usepackage[danish, english]{babel}
\usepackage{amssymb}
\usepackage{stmaryrd}
\usepackage{amsmath}
\usepackage{amsthm}
%\usepackage{eucal}
%\usepackage{subfigure}
\usepackage[pdftex]{color, graphicx}
\usepackage[final]{pdfpages}
\usepackage[final]{hyperref}

\usepackage{tabularx}
\usepackage{multicol}
\setlength{\vleftmargin}{0em}
\setlength{\columnsep}{30pt}
\usepackage[none]{hyphenat}


\usepackage{tikz}
\usepackage{todonotes}

\renewcommand{\chaptermark}[1]{\markboth{\thechapter\quad #1}{\thechapter\quad #1}}
\renewcommand{\sectionmark}[1]{\markright{\thesection\quad #1}}
\renewcommand{\tocmark}{\markboth{\contentsname}{\contentsname}}
\renewcommand{\bibmark}{\markboth{\bibname}{\bibname}}
\renewcommand{\glossarymark}{\markboth{\glossaryname}{\glossaryname}}
\renewcommand{\indexmark}{\markboth{\indexname}{\indexname}}

%\makepagestyle{suneNormal}
%\makeoddhead{suneNormal}{\scshape\rightmark}{}{\bfseries\thepage}
%\makeevenhead{suneNormal}{\bfseries\thepage}{}{\scshape\rightmark}
%\makeheadrule{suneNormal}{\textwidth}{\normalrulethickness}
%\pagestyle{suneNormal}
%\copypagestyle{suneChapter}{suneNormal}
%\makeoddhead{suneChapter}{\scshape\leftmark}{}{\bfseries\thepage}
%\makeevenhead{suneChapter}{\bfseries\thepage}{}{\scshape\leftmark}
%\aliaspagestyle{chapter}{suneChapter}

\makepagestyle{sangbog}
\makeoddfoot{sangbog}{}{}{\large\bfseries\thepage}%\makeoddfoot{sangbog}{}{}{\large\thepage}
\makeevenfoot{sangbog}{\large\bfseries\thepage}{}{}%\makeevenfoot{sangbog}{\large\thepage}{}{}
\copypagestyle{sangbogGreek}{sangbog}
\makeoddfoot{sangbogGreek}{}{}{\large\boldmath\pageAutoShift{\thepage}}%\makeoddfoot{sangbog}{}{}{\large\thepage}
\makeevenfoot{sangbogGreek}{\large\boldmath\pageAutoShift{\thepage}}{}{}%\makeevenfoot{sangbog}{\large\thepage}{}{}

\pagestyle{sangbogGreek}
\aliaspagestyle{chapter}{empty}


\makepagestyle{front}
\makeoddfoot{front}{}{}{\large\bfseries Ungdommens Naturvidenskabelige Forening}
\makeevenfoot{front}{}{}{\large\bfseries Ungdommens Naturvidenskabelige Forening}


\numberwithin{equation}{chapter}
\numberwithin{figure}{chapter}
\numberwithin{table}{chapter}
\frenchspacing

%\makeindex
%\makeglossary

%How the glossary looks
%\changeglossactual{+}
%\changeglossnum{\thepage}
\changeglossnumformat{|hyperpage}%% for hyperlinks

%Ugly hack so \colon survives into the glossary
%\renewcommand*{\begintheglossaryhook}{\newcommand{\colonn}{\colon}}

\newbox\tempboxa
\renewcommand{\glossitem}[4]{%
\sbox\tempboxa{#1 \quad #2 #3 #4}%
\par
\ifdim\wd\tempboxa<0.8\linewidth
#1 \quad #2 #3 \dotfill #4\relax
\else
\hangindent 2em
#1 \dotfill #4\\
#2 #3
%#1 \dotfill #4\\
%\parbox{\linewidth}{
%\leftskip 2em\rightskip 1em
%#2 #3
%}
\fi
}




% New commands
\newcommand{\e}[0]{$\varepsilon$}
\newcommand{\ph}[0]{$\varphi$}

\newcommand{\Z}[0]{\mathbb Z}
\newcommand{\N}[0]{\mathbb N}
\newcommand{\R}[0]{\mathbb R}
\newcommand{\Q}[0]{\mathbb Q}
\newcommand{\C}[0]{\mathbb C}

\newcommand{\lrepeat}{\ensuremath{\lVert : \,}}
\newcommand{\rrepeat}{\ensuremath{\, : \rVert}}


% TikZ for graphs and diagrams
\usetikzlibrary{arrows,matrix}
\tikzset{dot/.style={circle,fill=black,thick,inner sep=0pt,minimum size=1mm,draw}}
\tikzset{arrow/.style={semithick,>=stealth',shorten >=1pt,shorten <=1pt}}
%\tikzset{arrow/.style={semithick,>=to,shorten >=1pt,shorten <=1pt}}
\tikzset{equal/.style={arrow,double distance=2pt}}

%Enumerating with small roman integers as standard
\renewcommand{\theenumi}{(\roman{enumi})}\renewcommand{\labelenumi}{\theenumi}

\makeatletter
\newcommand*{\greek}[1]{%
  \expandafter\@greek\csname c@#1\endcsname
}
\newcommand*{\@greek}[1]{%
  $\ifcase#1
  \or\alpha\or\beta\or\gamma\or\delta\or\varepsilon
  \or\zeta\or\eta\or\theta\or\iota\or\kappa\or\lambda
  \or\mu\or\nu\or\xi\or o\or\pi\or\rho\or\sigma
  \or\tau\or\upsilon\or\varphi\or\chi\or\psi\or\omega
  \or\text{\bfseries \ae}\or\text{\bfseries \o}\or\text{\bfseries \aa}
  \or\mathfrak a\or\mathfrak b\or\mathfrak c\or\mathfrak d\or\mathfrak e
  \else\@ctrerr\fi$
}
\newcommand*{\Greek}[1]{%
  \expandafter\@Greek\csname c@#1\endcsname
}
\newcommand*{\@Greek}[1]{%
  \ifcase#1\or A\or B\or$\Gamma$\or$\Delta$\or E
  \or Z\or H\or$\Theta$\or I\or K\or$\Lambda$
  \or M\or N\or$\Xi$\or O\or$\Pi$\or P\or$\Sigma$
  \or T\or Y\or$\Phi$\or X\or$\Psi$\or$\Omega$
  \or$\mathfrak A$\or$\mathfrak B$\or$\mathfrak C$\or$\mathfrak D$
  \or$\mathfrak E$\or$\mathfrak F$\or$\mathfrak G$\or$\mathfrak H$
  \else\@ctrerr\fi
}

\newcommand*{\symcount}[1]{%
  \expandafter\@symcount\csname c@#1\endcsname
}
\newcommand*{\@symcount}[1]{%
  $\ifcase#1
  1\or2\or3
  \or 4\or 5\or6\or7\or 8\or 9 \or 11\or12\or13
  \or 14\or15\or 16\or17
  \or18
  \or 19\or 20
  \or 21\or 22
  \or 23
  \or24\or25\or 26\or 27
  \or28\or29\or30
  \or31\or32\or33\or34\or35\or36\or37\or38\or39\or40\or41\or42\or43\or44\or45\or46\or47\or48\or49\or50\or51\or52
  \else\@ctrerr\fi$
}

\newcommand*{\intcount}[1]{%
  \expandafter\@intcount\csname c@#1\endcsname
}
\newcommand*{\@intcount}[1]{%
  $\ifcase#1
  1\or2\or3
  \or 4\or 5\or6\or7\or 8\or 9
  \or10 \or 11\or12\or13
  \or 14\or15\or 16\or17
  \or18
  \or 19\or 20
  \or 21\or 22
  \or 23
  \or24\or25\or 26\or 27
  \or28\or29\or30
  \or31\or32\or33\or34\or35\or36\or37\or38\or39\or40\or41\or42\or43\or44\or45\or46\or47\or48\or49\or50\or51\or52
  \else\@ctrerr\fi$
}

\makeatother


\newcommand{\standardpage}{\arabic{page}}
\renewcommand{\thepage}{\standardpage}
\newcommand{\standardpoem}{\arabic{poem}}
\renewcommand{\thepoem}{\standardpoem}
%\setcounter{page}{0}


%Indikerer om Fulbert og Beatrice er passeret - så efterfølgende sidetal skal rykkes:
\newif\ifFulbert

\newcounter{value}
%Til udskrivning af sidetal på de enkelte sider (afhængigt af Fulbert og Beatrice):
\newcommand{\pageAutoShift}[1]{
    \setcounter{value}{#1}
    \ifFulbert
        \addtocounter{value}{-1}
    \fi
    %{\boldmath\greek{value}}
    {\boldmath\symcount{value}}
}
%Til udskrivning af sidetal i indeks (afhængigt af Fulbert og Beatrice):
\newcommand{\pageNoShift}[1]{
    \setcounter{value}{#1}
    %{\boldmath\greek{value}}
    {\boldmath\symcount{value}}
}
\newcommand{\pageWithShift}[1]{
    \setcounter{value}{#1}
    \addtocounter{value}{-1}
    %{\boldmath\greek{value}}
    {\boldmath\symcount{value}}
}

%Opretter sang i indeks - med korrekt sidetal (afhængigt af Fulbert og Beatrice).
%Syntaksen er følgende: \autoIndex[<alfabetisering>]{<sangtitel>}
\newcommand{\autoIndex}[2][]{
\ifthenelse{\equal{#1}{}}
{\autoIndexHelp{#2}{#2}}
{\autoIndexHelp{#1}{#2}}
}
\newcommand{\autoIndexHelp}[2]{
\ifFulbert
\index{#1@\protect\flagverse{\thepoem}#2|pageWithShift}
\else
\index{#1@\protect\flagverse{\thepoem}#2|pageNoShift}
\fi
}
\begin{comment}
\newcommand{\autoIndex}[2][{#2}]{
\ifFulbert
\index{{#1}@\protect\flagverse{\thepoem}{#2}|pageWithShift}
\else
\index{{#1}@\protect\flagverse{\thepoem}{#2}|pageNoShift}
\fi
}
\end{comment}



%\xindyindex
\makeindex
%\makeglossary
